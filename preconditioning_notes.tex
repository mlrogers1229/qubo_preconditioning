%% AMS-LaTeX Created with the Wolfram Language : www.wolfram.com

\documentclass{article}
\usepackage{amsmath, amssymb, graphics, setspace}

\newcommand{\mathsym}[1]{{}}
\newcommand{\unicode}[1]{{}}

\begin{document}

\title{Modifying QUBO Parameters to Improve Quantum Annealer Performance}
\author{Michael L Rogers}
\date{December 2, 2019}
\maketitle

Los Alamos National Laboratory, XCP-8

Los Alamos National Laboratory, CCS-3

Abstract



We consider the possibility of improving the performance of a superconducting quantum annealer (QA) by modifying the input parameters given to the
device. Here, we are focusing only on the problem of the loss of adiabatic evolution due to the occurance of very small spectral gaps { }during the
annealing time period. To this end, we propose modifying the input parameters, which specify the Hamiltonian applied near the end of the annealing
time, { }in a way that both: 1) Increses the average spectral gap over some range of final states with energies close to the ground state, i.e.,
reduces the spectral density there , and; 2) Retains the same qubit ground state vector, or at least keeps it within the bottom 1$\%$. { }



Whether or not this can be done in a way that significantly improves QA performance is undoubtedly problem dependent, asmany { }QUBO problems are
NP-complete, and it is not expected that this can be solved on quantum annealers. { }However many difficult optimizatio problems with only polynomial
complexity may sbe encoded QUBO problems and it may still be possible to demonstrate a quantum advantage with these problems,. By analyzing when
and if this may be possible, we hope to better quantify and understand the capabilities and limitations of QA machines.



We show how the user-input floating-point QUBO parameters constrain the the qubit vector states. This provides a simple relation which is useful
for developing strategies to modfy the parameters to increase the spectral gap. { }From these considerations we present three different strategies
for testing potentially increasing the spectral gap. { }Finally we test each of these strategies, in turn, applied to QUBO problems of varying compuatational
difficulty and present our results.

\section*{Motivation}

\subsection*{The Spectral Gap and Spectral Density}

Quantum annealing (QA) is a promising alternative paradigm for quantum computing, which has been shown to be, in theory, as powerful as more conventional
gated quantum computers, and that is theoretically well suited for solving hard optimization problems. { }It is also much easier to construct quantum
annealing with large numbers of superconducting qubits, and these qubits tend to be simpler and more fault tolerant than the superconducting gated
qubits. { }However, the QA approach to quantum computing has yet to demonstrate any quantum advantage on solving any problem relative to classical
optimizers. { }As commerical quantum annealers have become available in recent years, a number of difficulties have been discovered which are thought
to limit the ultimate computational capability of these machines. { }However the ultimate limitations are not yet fully understood or quantified.


One of most important limiting factors of QA machines, is existence of { }small gaps between neighboring energy levels encountered during the annealing
time evolution. For quantum annelers this is particularly problematic tf there is a sufficiently small gap between the ground state energy and first
excited state of the final state Hamiltonian. { }This causes the the time evolution to be non-adiabatic, resulting in dispersion of the state vector
as it evolves toward the final ground state. When this happens, { }the qubit ground state vector { }will frequently mix with a large number of other
qubit states nearby in enrergy, { }In many problems this will make it difficult or impossible to find the true ground state in a reasonable amount
of time. This problem is unavoidable for quantum annealing with a large number of qubits, \(N\), as the number of final states increases in propottion
to \(2^N\) while the energy range only increases in proportion to \(N\). { }Therefore, for arbitrary QA final states, the average spectral density
goes as \(2^{-N}.\) { }So, for typical large problems, the spectral gap approaches zero exponentially.

It is the objective of the { }present report to consider whether it is possible to obtain better QA performance by simply modying the QUBO parameters
which are used to {``}program{''} the QA. { }We also present a simple relation which clearly shows how the input parameters constrain the total energy
surface for the qubit state vectors. { }This will prove useful, in general, for analyzing and quantifying the capabilities and limitations of the
QA approach to quantum computing. { }Since we are only considering the spectral gap problem here, we assume that the QA machine is evolving quantum
mechanically during the annealing process, but that the ground state is not evolving adiabatically due to very closely spaced low energy states near
the end of the annealing run. 

\subsection*{Quantum Annealing and Adiabatic Evolution}

The standard quantum annealing algorthm with superconducting qubits works by time evolving the applied Hamiltionian,starting with an initial Hamiltonian,
\(H_0\), corresponding to an applied magnetic field paralell to the chip, which puts the { }qubits in a randomized and uncorrelated initial state,
to the applied final Hamiltonian, \(H_1\), which is specified by the user in the programming interface. The system is then measured at some annealing
time, \(\tau _a\) at which time the Hamiltonian approximates the time-independent \(H_1\). If the system has evolved entirely adiabatically from
the initial ground state over the entire time \(\tau _{0 }\) to \(\tau _a\), it will be measured to be in the ground state of \(H_1\), or very nearly
so. { }If it is not, and the annealing time were large enough, more annealing runs using the same QUBO parameters should eventually measure the correct
the ground state, however this may take an exponentially long time for sufficiently hard problems..

The time-dependent Hamiltonian during the annealing process is given by,

\(\hat{H}(t) = (1-g(t) )\hat{H}_0 + g(t) \hat{H}_1\),

where the function g(t) is a mononically increaseing functions of time.

At t=0, { }the energy levels start out completely degenerate, but would evolve adiabatically to remain in the instantaneous ground state during the
annealing time, if they energy levels near the instatnaneous ground state became sufifciently well separated well before \(\tau _a\), and remaind
so. { }But, if { }energy levels near the instantaneous ground state become too close together during this time, they ground state will mix with these
levels, destroying the adiabatic evolution of the ground state. However, since the evolution from \(H_0\) to \(H_1\)is monotonic, we would expect
that after the initial separation the energy levels near the ground state would not get too small unless the energy levels are already too small
for \(H_1\). { }So, we expect that by increasing the spectral gap for \(H_1\)that we can improve the adiabatic behavior, create less mixing with
neighboring states, and therefore have a higher probability of measuring the ground state.

\section*{Modifying QUBO Parameters to Increase the Spectral Gap}

\subsection*{How the QUBO Parameters Constrain the Hamiltonian Spectrum}

The applied final Hamiltonan,, \(H_1\), for { }N qubits is given by,

\(\hat{H} =\sum _{i=1}^N \sum _{j\neq i} \hat{q_i}{}^TB_{\text{ij}} \hat{q_i} +\sum _{i=1}^N \hat{q_i}{}^T \hat{a}_i\)

where , B is the symmetric form (in our convenction) of the off-diagonal QUBO {``}B{''} matrix of parameters, and $\mathit{a}$ is the QUBO {``}a{''}
vector. The operators \(\hat{q_i}\) are the individual qubit operators on n qubit product states, \(|\overset{\rightharpoonup }{q}_i >_\) such that
the eigenvalues of individual qubit states satsify,

\(\hat{q}_i\text{  }| q_{i }> = q_{i }| q_{i }>\),

where \(q_{i }\in \{0,1\}\) .

The equation in terms of the eigenvalues is equivalent to the follwoing objective function for an arbiatrary real vector, \(\overset{\rightharpoonup
}{x}\), subject to constraint that\(x_{i }=q_i\in \{0,1\}\).

\(H \approx  \overset{\rightharpoonup }{x}^TB \overset{\rightharpoonup }{x} + \overset{\rightharpoonup }{x}^T \overset{\rightharpoonup }{a},\)

We are employing a notation introduced by { }Dirac in his work on constrained forms of dynamic in which the sign{''} $\approx ${''} is used to mean
{``}equal when the constraint is applied{''}. In terms of this notation, we may then also write this as,.

\(H \approx \text{  }\overset{\rightharpoonup }{x}^TQ\text{  }\overset{\rightharpoonup }{x} ,\)

\(\text{where},\)

\(\text{  }Q \equiv  B+\text{  }A\)

and cp his the diagonal matrix with the elements of $\mathit{a}$ on the diagonal.

If the system is measure in an energy eigenstate, \(E_i\) of \(\hat{H}\), { }we may write the energy in terms of the eigenvalue array \(\hat{q}_i\)
as,

\(E_i = \overset{\rightharpoonup }{q}_i{}^T\cdot Q\cdot  \overset{\rightharpoonup }{q}_i .\)

To understand how the QUBO parameters constrain the quantum Hamiltonian spectrum, consider the (classical) spectrum of the matrix, Q.. { }Assuming
the paramers $\mathcal{B}$ and $\mathit{a}$ are given, for a particular problem, first form Q and determine its eigenvalues and eigenvectors. { }This
will be always possible because Q is a real, symmetric matrix, and so it it is Hermitian. Also, the operations are floating-point matrix operations,
which are computable in no more than \(o\left(N^3\right)\) steps. { }Write $\mathit{k}$-th the eigenvectorsof Q as \(\hat{v}_k\). Then we have

\(\text{  }Q\hat{v}_k= \lambda _k\hat{v}_k.\)

Since the \(\hat{v}_k\) form a complete orthogonal basis in the space \(\mathbb{R}^N\)(where $\mathcal{N}$ is the number of qubits). we may expand
any vector in this space in this basis, i.e.,

\(\overset{\rightharpoonup }{x} = \sum _{k=1}^N  \left(\hat{v}_k\cdot \overset{\rightharpoonup }{x}\right) \hat{v}_k.\)

In particular, we may also do this for the vectors of qubit eigenvalues, since they are also real-valued vectors in this space, although they are
constrained to have values of 0 or 1. Thus, we may write,

\(\overset{\rightharpoonup }{q}_i = \sum _{k=1}^N  \left(\hat{v}_k\cdot \overset{\rightharpoonup }{q}_i\right) \hat{v}_k\)

But, since the eigenvectors \(\hat{v}_k\) are orthonormal, we can substitude this into the energy equation to find,

\(E_i= \sum _{k=1}^N  \left\left| \hat{v}_k\cdot \overset{\rightharpoonup }{q}_i\right\right| {}^2\lambda _k.\)

From this we see that we may also express the energy of an eigenvalue of the Hamiltonian on the space of qubits as a sum of contributions from each
of the eigenvalues of $\mathcal{Q}$, weighted by the the inner product, of \(\hat{v}_k\) and \(\overset{\rightharpoonup }{q}_i\) squared. We may
not obtain any energy eigenstate\(\overset{\rightharpoonup }{q}_i\) this way, bu this relatiot shows the QUBO parameter constrain the energy spectrum,
given the eigenstate. { }With this expression we may also express the gap, \(\Delta _i\),between any QUBO energy and the ground state energy takes
the form,

\(\Delta _{\text{i0}} \equiv  E_i-E_0= \sum _{k=1}^N  \left(\left\left| \hat{v}_k\cdot \overset{\rightharpoonup }{q}_i\right\right| {}^2-\left\left|
\hat{v}_k\cdot \overset{\rightharpoonup }{q}_0\right\right| {}^2\right)\text{  }\lambda _k\)

\(\text{                                     }=\sum _{k=1}^N \left( \hat{v}_k\cdot \left(\overset{\rightharpoonup }{q}_i+\overset{\rightharpoonup
}{ q}_0 \right)\right) \left( \hat{v}_k\cdot \left(\overset{\rightharpoonup }{q}_i-\overset{\rightharpoonup }{ q}_0 \right)\right)\text{   }\lambda
_k\)

This equation shows that each energy gap is determined by both the spectrum of $\mathcal{Q}$ and the the square of the difference between the inner
products of \(\overset{\rightharpoonup }{q}_i\) and \(\overset{\rightharpoonup }{q}_0\) with each \(\hat{v}_k\). So we can see that there are two
natural but opposite approaches for modifying $\mathcal{Q}$ to change the spectral gap. We may keep the eigenvevtors constant, while changing the
eigenvalues, or we may keep the eigenvalues constant while changing the eigenvectors. { } There is also a third way, which involves introducing small
perturbations to the Q matrix via the vector of weights. We will examine how these methods to see how it may be possible to increase the spectral
gap in the following sections.

\subsection*{Methods to Decrease the Spectral Density}

The Spectral Scaling Method

This idea for this method is to change $\mathcal{Q}$ to some new $\mathcal{Q}${'} so as to keep the same eigenvectors \(\hat{v}_k\) but change some
or all of the eigenvalues, \(\lambda _k.\) This can easily be accomplished by using the \(\hat{v}_k\) to construct projection operators, which, when
multiplied by a new eigenvalue { }\(\lambda '_k\), and added togeher yields a new matrix, $\mathcal{Q}$ { }, with the that but with the same eigenvectors
and overlaps with qubit states. { } Tochange every eigenvalue to \(\lambda \to \lambda _k'\), { }we would simply define, { } { } { } { } 

$\mathcal{Q}$' = { }\(\sum _{k=1}^N \lambda '_k \hat{v\cdot } \hat{v}_k{}^T\).

The result of this transformation would just changes the spectrum to something with a lower spectral density, however the corresponding eigenvectors
of $\mathcal{Q}${'} will be unchanged. This approach will change the spectral gap only by changing the eignvalues.

One very effective approach has been to scale the eigenvalues matrix by adding a contribution proportional to each eigenvalue, so that a quadratic
spreading occurs, which preferntially weights those eigenvalues with higher absoltue values. Other kinds of scaling or scaling plus translation operations
on the original set of eigenvalues have been tested with some degree of success.

The Similary Tranform Method

In this approach, we change the part of the spectral gap equation in brackets, i.e., we shift the spectral gap by changing the {``}overlaps{''} or
dot products between { }\(\hat{v}_k\) and \(\overset{\rightharpoonup }{q}_i\), while keeping the eigenvalues invariant. { }This is accomplished by
simply applying a similarity transformation to $\mathcal{Q}$. { }We define,

$\mathcal{Q}${'} = \(R^{-1}\mathcal{Q} R\),

where, R is a unitary matrix, essentially a rotation matrix in \(\mathbb{R}^N\). This will keep the spectrum invariant (i.e., the eigenvalues), but
it will effectively rotate the eigenvectors by R. { }This will change the overlaps in the spectral gap equation, which can change the spectral gap.


This is being investigated, but currently we do not know of a systematic approach to chose the rotations.

The Perturbative Method

This approach involves adding a diagonal real matrix (which is also Hermitian) { }to $\mathcal{Q}$ { }which is a {``}small{''} perturbation in the
sense that its matrix norm is small relative to the norm of $\mathcal{Q}$. { }Then, both the eigenvalues and eigenvectors of $\mathcal{Q}${`} will
be changed, however, none of the eigenvalues will change in their order, and the eigenvalues will be constrained to satisfy the Weyl Inequalities.
{ }

This method has been observed to yield better spectral densities by trial-and-error, but no systematic approach has been found to accomplish this
yet.

\end{document}
